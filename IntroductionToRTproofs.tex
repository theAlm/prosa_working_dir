% This a file form my teaching Coq project.
% it belongs to IntroductionToCoq.tex
% ignore it in the PROSA repository


\subsection{Introduction to RT-Proofs}
% What is RT-proofs?
% Probably the PROSA-VM  was downloaded on the 25.11.2019 by Kai Beckmann.
This section contains listings from the working directory found at \url{git@gitlab.cs.hs-rm.de:almeroth/prosa_working_dir.git}.
They show the most important and modified parts of the Coq-code form the RT-Proofs clone found in this repository.\\
The clone is from the artifact evaluation form this work \cite{PROSA_schedubility_analysis} an can be found here \cite{PROSA_ECRTS_Artifact}.
Note that the style definitions for Coq-code from \cite{CoqStyleListing} were used in the listings of this work to pretify this {\LaTeX}-document.
Formal Proofs for real-time systems (RT-Proofs) is a project running between multiple research faculties.
The PROSA library is where this development takes place. The output of the RT-Proofs project  is published here \url{https://prosa.mpi-sws.org/}.

%-------------------------------------------------------------------------------
% Could not make this work.
% It should be maintanable checking the documentation of the CTAN package gitinfo

%\begin{tabular}[t]{rl}
%			Erstellt von: & {\gitAuthorName}\\
%                       Zuletzt bearbeitet von: & {\gitCommitterName}\\
%			Email: & {\gitAuthorEmail}\\
%			Erste Version vollendet: & {\writtendate}\\
%			Version: & {\gitAbbrevHash}\\
%			Letzte Änderung: & {\gitAuthorIsoDate}\\
%\end{tabular}
%--------------------------------------------------------------------------------

\paragraph{The Repository}

This repository is sought to be ment for self-teaching somebody to work with the proof assistant Coq and the PROSA -Frame-Work.


\includebashcode{README.md}

% Please forgive my breavity.
